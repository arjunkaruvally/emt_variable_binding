\documentclass[11pt]{article}
\usepackage{fullpage}
\usepackage{amsmath,amsfonts,amsthm,amssymb}
\usepackage{newtxtext}
\usepackage{newtxmath}
\usepackage{listings}
\usepackage{url}
\usepackage{cancel}
\usepackage{algorithm}
\usepackage{graphicx}
\usepackage{caption} 
\usepackage{algpseudocode}
\usepackage{bbm}
\usepackage{float}
\usepackage{framed}
\usepackage{enumerate}
\usepackage{color}
\usepackage{physics}
\usepackage{empheq}
\usepackage{caption}
\usepackage{subcaption}
\usepackage{breqn}
\usepackage{multirow}
\usepackage{pdfpages}

\usepackage{tikz}
\tikzset{elegant/.style={smooth, black, thick, samples=101}}
\tikzstyle{round}=[circle ,text centered,draw=black]
\tikzstyle{arrow} = [-,>=stealth]
\tikzstyle{rct}=[rectangle,draw,thin,fill=white]
\usetikzlibrary{arrows,decorations.pathmorphing,backgrounds,positioning,fit,matrix,calc}

\usepackage[most]{tcolorbox}

\newtcbox{\mymath}[1][]{%
    nobeforeafter, math upper, tcbox raise base,
    enhanced, colframe=blue!30!black,
    colback=blue!30, boxrule=1pt,
    #1}

\usepackage[colorlinks=true, linkcolor=red, urlcolor=blue, citecolor=blue]{hyperref}
\allowdisplaybreaks


\DeclareMathOperator*{\E}{\mathbb{E}}
\DeclareMathOperator{\X}{\mathbf{X}}
\DeclareMathOperator{\mean}{\boldsymbol{\mu}}
\DeclareMathOperator{\std}{\boldsymbol{\sigma}}
\let\Pr\relax
\DeclareMathOperator*{\Pr}{\mathbb{P}}
\DeclareMathOperator*{\R}{\mathbb{E}}
\newcommand\thres{{\mbox{\it{thres}}}}

\newtheorem{theorem}{Theorem}[section]
\newtheorem{corollary}{Corollary}[theorem]
\newtheorem{lemma}[theorem]{Lemma}
\theoremstyle{definition}
\newtheorem{definition}{Definition}[section]

%New colors defined below
\definecolor{codegreen}{rgb}{0,0.6,0}
\definecolor{codegray}{rgb}{0.5,0.5,0.5}
\definecolor{codepurple}{rgb}{0.58,0,0.82}
\definecolor{backcolour}{rgb}{0.95,0.95,0.92}

%Code listing style named "mystyle"
\lstdefinestyle{mystyle}{
  backgroundcolor=\color{backcolour},   commentstyle=\color{codegreen},
  keywordstyle=\color{magenta},
  numberstyle=\tiny\color{codegray},
  stringstyle=\color{codepurple},
  basicstyle=\ttfamily\footnotesize,
  breakatwhitespace=false,         
  breaklines=true,                 
  captionpos=b,                    
  keepspaces=true,                 
  numbers=left,                    
  numbersep=5pt,                  
  showspaces=false,                
  showstringspaces=false,
  showtabs=false,                  
  tabsize=2
}

%"mystyle" code listing set
\lstset{style=mystyle}

\topmargin 0pt
\advance \topmargin by -\headheight
\advance \topmargin by -\headsep
\textheight 8.9in
\oddsidemargin 0pt
\evensidemargin \oddsidemargin
\marginparwidth 0.5in
\textwidth 6.5in

\parindent 0in
\parskip 1.5ex

\begin{document}

	\begin{center}
		\framebox{
			\vbox{
				\hbox to 6.50in { {\bf Episodic Memory Theory of RNNs} \hfill }
				\vspace{4mm}
				\hbox to 6.50in { {\Large \hfill Energy Analysis \hfill} }
				\vspace{2mm}
				\hbox to 6.50in { { Arjun Karuvally \hfill} }
			}
		}
	\end{center}
	\vspace*{1mm}
	\section{System - Derivation from Continuous Time Dynamics}
	
The general system of neurons is governed by the following equations. The state variables of the dynamical system are $V_f \in \mathbb{R}^{N_f \times 1}, V_h \in \mathbb{R}^{N_h \times 1}, V_d \in \mathbb{R}^{N_f \times 1}$. The interactions are represented by $\Xi \in \mathbb{R}^{N_f \times N_h}$ and $\Phi \in \mathbb{R}^{N_h \times N_h}$. $\Phi$ represents synaptic strength \textit{from} $V_h$ \textit{to} $V_h$.
	
\begin{empheq}{equation}
\begin{dcases}
	\mathcal{T}_f \dv{V_f}{t} =& \sqrt{\alpha_s} \, \Xi \, \sigma_h(V_h) - V_f \, , \\
	\mathcal{T}_h \dv{V_h}{t} =& \sqrt{\alpha_s} \, \Xi^\top \, \sigma_f(V_f) + \alpha_c \Phi^\top \Xi^\top V_{d} - V_h \, , \\
	\mathcal{T}_d \dv{V_d}{t} =& \sigma_f(V_f) - V_d \, .
\end{dcases}
\end{empheq}
%
The system has an energy function given by
%
\begin{dmath}
	E = \Bigg[ V_f^\top \, \sigma_f(V_f) - L_f\Bigg] + \Bigg[ V_h^\top \, \sigma_h(V_h) - L_h \Bigg] - \Bigg[ \sqrt{\alpha_s} \, \sigma_f(V_f)^\top \, \Xi \, \sigma_h(V_h) \Bigg] - \alpha_c \Bigg[ V_{d}^\top \, \Xi \, \Phi \sigma_h(V_h)  \Bigg]
\end{dmath}
%
Conditions:
\begin{itemize}
	\item $\mathcal{T}_h \rightarrow 0$
	\item $\mathcal{T}_d \rightarrow 0$
	\item discretize time for $V_f$
	\item $\sigma_h(X) = X$
	\item $\alpha_s = \alpha_c = 1$
	\item $\mathcal{T}_f = 1$
\end{itemize}    
%
$\sigma_h(X) = X \implies L_h = \frac{1}{2} \, V_h^\top \, V_h$

\section{Deriving governing equations}

From a given time $t$, the update equations are given as

\begin{empheq}{equation}
\begin{dcases}
	\mathcal{T}_f (V_f(t+1) - V_f(t)) =& \Xi \, \sigma_h(V_h(t)) - V_f(t) \, , \\
	V_h(t) =& \Xi^\top \, \sigma_f(V_f(t)) + \Phi^\top \Xi^\top V_{d}(t) \, , \\
	V_d(t) =& \sigma_f(V_f(t)) \, .
\end{dcases}
\end{empheq}
%
\begin{empheq}{equation}
\begin{dcases}
	\mathcal{T}_f (V_f(t+1) - V_f(t)) =& \Xi \, \sigma_h(V_h) - V_f(t) \, , \\
	V_h(t) =& \Xi^\top \, \sigma_f(V_f(t)) + \Phi^\top \Xi^\top \sigma_f(V_f) \, , \\
\end{dcases}
\end{empheq}
%
\begin{empheq}{equation}
\begin{dcases}
	\mathcal{T}_f (V_f(t+1) - V_f(t)) =& \Xi \, V_h - V_f(t) \, , \\
	V_h(t) =& (I + \Phi^\top) \Xi^\top \sigma_f(V_f) \, , \\
\end{dcases}
\end{empheq}
%
\begin{dmath}
	\mathcal{T}_f (V_f(t+1) - V_f(t)) = \Xi \, (I + \Phi^\top) \Xi^\top \sigma_f(V_f) - V_f(t)
\end{dmath}
%
Final discrete upate equation
%
\begin{dmath}
	V_f(t+1) = \Xi \, (I + \Phi^\top) \Xi^\top \sigma_f(V_f)
	\label{governingDynamics:original}
\end{dmath}
%
Restrict the norm of matrix $||\Xi \, (I + \Phi^\top) \Xi^\top|| \leq 1$. \\
%
This allows us to consider the transformation $V'_f = \sigma_f(V_f)$, so for invertible $\sigma_f$, 
%
\begin{dmath}
	\sigma^{-1}_f(V'_f(t+1)) = \Xi \, (I + \Phi^\top) \Xi^\top V'_f
\end{dmath}
%
\begin{dmath}
	\sigma^{-1}_f(V'_f(t+1)) = \Xi \, (I + \Phi^\top) \Xi^\top V'_f
\end{dmath}
%
\begin{dmath}
	V'_f(t+1) = \sigma_f(\Xi \, (I + \Phi^\top) \Xi^\top V'_f)
	\label{governingDynamics:rnn}
\end{dmath}
%
this is a general update equation for an RNN without bias.
%
\section{Topological Conjugacy with RNNs}
%
Proof that dynamical systems governed by Equations \ref{governingDynamics:original} and \ref{governingDynamics:rnn} are topological conjugates.

Consider $f(x) = \Xi \, (I + \Phi^\top) \Xi^\top \sigma_f(x)$ for Equation \ref{governingDynamics:original} and $g(x) = \sigma_f(\Xi \, (I + \Phi^\top) \Xi^\top x)$ for Equation \ref{governingDynamics:rnn}. Consider a homeomorphism $h(y) = \sigma_f(y)$ on $g$. Then,
%
\begin{dmath}
	(h^{-1} \circ g \circ h) (x) = \sigma_f^{-1}( \sigma_f(\Xi \, (I + \Phi^\top) \Xi^\top \sigma_f(x)) ) = \Xi \, (I + \Phi^\top) \Xi^\top \sigma_f(x) = f(x)
\end{dmath}
%
So, for the homeomorphism $h$ on $g$, we get that $h^{-1} \circ g \circ h = f$ proving that $f$ and $g$ are topological conjugates. Therefore all dynamical properties of $f$ and $g$ are shared.

\section{Deriving Energy}
%
From the proof of topological conjugacy, we can show energy properties on only $f$.
%
Careful about the time in discretization. Keep $V_d$ as is only to convert in the end
%
\begin{dmath}
	E(t) = \Bigg[ V_f(t)^\top \, \sigma_f(V_f(t)) - L_f(t)\Bigg] + \frac{1}{2} V_h(t)^\top \, V_h(t) - \Bigg[ \sigma_f(V_f(t))^\top \, \Xi \, V_h(t) \Bigg] - \Bigg[ V_d(t)^\top \, \Xi \, \Phi V_h(t)  \Bigg]
\end{dmath}
%
\begin{dmath}
	E(t+1) = \Bigg[ V_f(t+1)^\top \, \sigma_f(V_f(t+1)) - L_f(t+1)\Bigg] + \frac{1}{2} V_h(t+1)^\top \, V_h(t+1) - \Bigg[ \sigma_f(V_f(t+1))^\top \, \Xi \, V_h(t+1) \Bigg] - \Bigg[ V_d(t+1)^\top \, \Xi \, \Phi V_h(t+1)  \Bigg]
\end{dmath}
%
Working with $V_h(t) = \Xi^\top \sigma_f(V_f) + \Phi^\top \Xi^\top V_d(t)$. The next subsections will try to compute $E(t+1) - E(t)$.
%
Instead of directly discretizing $E$, lets try to use the continuous formulation first using $V_h = \Xi^\top \, \sigma_f(V_f) + \Phi^\top \, \Xi^\top V_{d}$
%
\begin{dmath}
	E(t) = \Bigg[ V_f^\top \, \sigma_f(V_f) - L_f\Bigg] + \frac{1}{2} V_h^\top \, V_h - \Bigg[ \sigma_f(V_f)^\top \, \Xi \, V_h \Bigg] - \Bigg[ V_d^\top \, \Xi \, \Phi V_h \Bigg]
\end{dmath}
%
\noindent\rule{8cm}{0.4pt} %%%%%%%%%%%%%%%%%%%%%%%%%%%%%%%%%%%%%
%
\begin{dmath}
	E_2(t) = \frac{1}{2} \left[ \left( \sigma_f(V_f)^\top \Xi + V_d^\top \Xi \Phi \right) \left( \Xi^\top \sigma_f(V_f) + \Phi^\top \Xi^\top V_d \right) \right]
\end{dmath}
%
\begin{dmath}
	E_2(t) = \frac{1}{2} \left[ \sigma_f(V_f)^\top \Xi \Xi^\top \sigma_f(V_f) + \textcolor{red}{\sigma_f(V_f)^\top \Xi \Phi^\top \Xi^\top V_d} + \textcolor{red}{V_d^\top \Xi \Phi \Xi^\top \sigma_f(V_f)} + V_d^\top \Xi \Phi \Phi^\top \Xi^\top V_d \right]
\end{dmath}
%
\begin{dmath}
	E_2(t) = \frac{1}{2} \textcolor{blue}{\sigma_f(V_f)^\top \Xi \Xi^\top \sigma_f(V_f)} + \textcolor{red}{\sigma_f(V_f)^\top \Xi \Phi^\top \Xi^\top V_d} + \frac{1}{2} V_d^\top \Xi \Phi \Phi^\top \Xi^\top V_d
\end{dmath}
%
\noindent\rule{8cm}{0.4pt} %%%%%%%%%%%%%%%%%%%%%%%%%%%%%%%%%%
%
\begin{dmath}
	E_3(t) = - \sigma_f(V_f)^\top \, \Xi \, V_h
\end{dmath}
%
\begin{dmath}
	E_3(t) = - \sigma_f(V_f)^\top \, \Xi \, \left( \Xi^\top \, \sigma_f(V_f) + \Phi^\top \, \Xi^\top V_{d} \right)
\end{dmath}
%
\begin{dmath}
	E_3(t) = - \left[ \textcolor{blue}{\sigma_f(V_f)^\top \, \Xi \, \Xi^\top \, \sigma_f(V_f)} + \textcolor{red}{\sigma_f(V_f)^\top \, \Xi \Phi^\top \, \Xi^\top V_{d}} \right]
\end{dmath}
%
\noindent\rule{8cm}{0.4pt} %%%%%%%%%%%%%%%%%%%%%%%%%%%%%%%%
%
\begin{dmath}
	E_4(t) = - \left[ V_d^\top \Xi \Phi \left( \Xi^\top \, \sigma_f(V_f) + \Phi^\top \, \Xi^\top V_{d} \right) \right]
\end{dmath}
%
\begin{dmath}
	E_4(t) = - \left[ V_d^\top \Xi \Phi \Xi^\top \, \sigma_f(V_f) + V_d^\top \Xi \Phi \Phi^\top \, \Xi^\top V_{d} \right]
\end{dmath}
%
\begin{dmath}
	E_4(t) = - \left[ \textcolor{red}{\sigma_f(V_f) \, \Xi \Phi^\top \Xi^\top V_d} + V_d^\top \Xi \Phi \Phi^\top \, \Xi^\top V_{d} \right]
\end{dmath}
%
\noindent\rule{8cm}{0.4pt} %%%%%%%%%%%%%%%%%%%%%%%%%%%%%%%%
%
\begin{dmath}
	E = E_1 + E_2 + E_3 + E_4
\end{dmath}
%
Simplifying the colored terms
%
\begin{dmath}
	E = V_f^\top \sigma_f(V_f) - L_f - \frac{1}{2} \textcolor{blue}{\sigma_f(V_f)^\top \, \Xi \, \Xi^\top \, \sigma_f(V_f)} - \textcolor{red}{\sigma_f(V_f)^\top \, \Xi \Phi^\top \Xi^\top V_d} -\frac{1}{2} V_d^\top \Xi \Phi \Phi^\top \, \Xi^\top V_{d}
\end{dmath}
%
\begin{empheq}[box=\tcbhighmath]{equation}
	E = V_f^\top \sigma_f(V_f) - \left[ L_f + \frac{1}{2} \sigma_f(V_f)^\top \, \Xi \, \Xi^\top \, \sigma_f(V_f) + \sigma_f(V_f)^\top \, \Xi \Phi^\top \Xi^\top V_d + \frac{1}{2} V_d^\top \Xi \Phi \Phi^\top \, \Xi^\top V_{d} \right]
\end{empheq}

\section*{Discrete Energy Update - Not complete (non linear terms issue)}

\subsection{Trial 1: Adiabatic $V_d$}
%
Case of direct substitution of time.
%
\begin{empheq}{equation}
\begin{dcases}
	V_f(t+1) =& \Xi \, \Xi^\top \, \sigma_f(V_f(t)) + \Xi \, \Phi^\top \, \Xi^\top V_{d}(t) , \\
	V_d(t) =& \sigma_f(V_f(t)) \, .
\end{dcases}
\end{empheq}
%
Consider the discrete energy equation at time $t$
%
\begin{dmath}
	E(t) = V_f(t)^\top \sigma_f(V_f(t)) - L_f(t) + \frac{1}{2} \sigma_f(V_f(t))^\top \, \Xi \, \Xi^\top \, \sigma_f(V_f(t)) - \sigma_f(V_f(t)) \, \Xi \Phi^\top \Xi^\top V_d(t) - \textcolor{red}{\frac{1}{2} V_d(t)^\top \Xi \Phi \Phi^\top \, \Xi^\top V_{d}(t)}
\end{dmath}
%
and 
%
\begin{dmath}
	E(t+1) = V_f(t+1)^\top \sigma_f(V_f(t+1)) - L_f(t+1) + \frac{1}{2} \sigma_f(V_f(t+1))^\top \, \Xi \, \Xi^\top \, \sigma_f(V_f(t+1)) - \sigma_f(V_f(t+1)) \, \Xi \Phi^\top \Xi^\top V_d(t) - \textcolor{red}{\frac{1}{2} V_d(t)^\top \Xi \Phi \Phi^\top \, \Xi^\top V_{d}(t)}
\end{dmath}
%
\subsubsection{E(t+1) in terms of V_f(t)}

\begin{dmath}
	E_1(t+1) = \left( \sigma_f(V_f(t))^\top \, \Xi \, \Xi^\top + V_{d}^\top(t) \, \Xi \, \Phi \, \Xi^\top  \right) \sigma_f(V_f(t+1))
\end{dmath}

\noindent\rule{8cm}{0.4pt} %%%%%%%%%%%%%%%%%%%%%%%%%%%%%%%% \\

\begin{dmath}
	E_3(t+1) = \left( \sigma_f(V_f(t))^\top \, \Xi \, \Xi^\top + V_{d}^\top(t) \, \Xi \, \Phi \, \Xi^\top  \right) \sigma_f(V_f(t+1))
\end{dmath}

\noindent\rule{8cm}{0.4pt} %%%%%%%%%%%%%%%%%%%%%%%%%%%%%%%% \\

How to deal with these nonlinear terms?

\section{Tanh Case - only Elmann RNN w/o bias - nonlinear terms issue}
%
\begin{dmath}
	E(t) = V_f(t)^\top \tanh(V_f(t))^\top - \sum \log |\cosh(V_f(t))| - \frac{1}{2} \tanh(V_f(t))^\top \, \Xi \, \Xi^\top \, \sigma_f(V_f(t)) - \tanh(V_f(t)) \, \Xi \Phi^\top \Xi^\top V_d(t) - \textcolor{red}{\frac{1}{2} V_d(t)^\top \Xi \Phi \Phi^\top \, \Xi^\top V_{d}(t)}
\end{dmath}
%
\begin{dmath}
	E(t+1) = V_f(t+1)^\top \tanh(V_f(t+1))^\top - \sum \log |\cosh(V_f(t+1))| - \frac{1}{2} \tanh(V_f(t+1))^\top \, \Xi \, \Xi^\top \, \sigma_f(V_f(t+1)) - \tanh(V_f(t+1)) \, \Xi \Phi^\top \Xi^\top V_d(t+1) - \textcolor{red}{\frac{1}{2} V_d(t+1)^\top \Xi \Phi \Phi^\top \, \Xi^\top V_{d}(t+1)}
\end{dmath}
%
\noindent\rule{8cm}{0.4pt} %%%%%%%%%%%%%%%%%%%%%%%%%%%%%%%% \\
%
\begin{dmath}
	\Delta E_1 = \tanh(V_f(t))^\top \, \Xi \, \Xi^\top \tanh(V_f(t+1)) + \textcolor{red}{V_{d}(t)^\top \Xi \, \Phi \, \Xi^\top \tanh(V_f(t+1))} - V_f(t)^\top \tanh(V_f(t)) 
\end{dmath}
%
\noindent\rule{8cm}{0.4pt} %%%%%%%%%%%%%%%%%%%%%%%%%%%%%%%% \\
%
\begin{dmath}
	\Delta E_2 = \sum \log \frac{|\cosh(V_f(t+1))|}{|\cosh(V_f(t))|} 
\end{dmath}
%
\noindent\rule{8cm}{0.4pt} %%%%%%%%%%%%%%%%%%%%%%%%%%%%%%%% \\
%
\begin{dmath}
	\Delta E_3 = - \frac{1}{2} \left[ \tanh(V_f(t+1))^\top \Xi \Xi^\top \tanh(V_f(t+1)) - \tanh(V_f(t))^\top \Xi \Xi^\top \tanh(V_f(t)) \right]
\end{dmath}
%
\noindent\rule{8cm}{0.4pt} %%%%%%%%%%%%%%%%%%%%%%%%%%%%%%%% \\
%
\begin{dmath}
	\Delta E_4 = - \left[ \tanh(V_f(t+1))^\top \, \Xi \Phi^\top \Xi^\top V_d(t+1) - \tanh(V_f(t))^\top \, \Xi \Phi^\top \Xi^\top V_d(t) \right]
\end{dmath}
%
Suppose adiabatic, then $V_d(t+1) = V_d(t)$
%
\begin{dmath}
	\Delta E_4 = - \left[ \tanh(V_f(t+1)) - \tanh(V_f(t)) \right]^\top \, \Xi \Phi^\top \Xi^\top V_d(t)
\end{dmath}
%
Trying simplification using identities
%
\begin{dmath}
	\Delta E_4 = - \left[ \tanh(V_f(t+1) - V_f(t))(1 - \tanh(V_f(t+1)) \tanh(V_f(t))) \right] \, \Xi \Phi^\top \Xi^\top V_d(t)
\end{dmath}
%
Suppose $V_d(t+1) = V_d(t) = \sigma_f(V_f(t))$
\begin{dmath}
	\Delta E_4 = - \left[ \textcolor{red}{\tanh(V_f(t+1))^\top \Xi \Phi^\top \Xi^\top \tanh(V_f(t))} - \textcolor{blue}{\tanh(V_f(t))^\top \Xi \Phi^\top \Xi^\top \tanh(V_f(t))} \right]
\end{dmath}
%
\noindent\rule{8cm}{0.4pt} %%%%%%%%%%%%%%%%%%%%%%%%%%%%%%%% \\
%
\begin{dmath}
	\Delta E = \Delta E_1 + \Delta E_2 + \Delta E_3 + \Delta E_4 + \Delta E_5
\end{dmath}
%
\begin{dmath}
	= \tanh(V_f(t))^\top \, \Xi \, \Xi^\top \tanh(V_f(t+1)) - V_f(t)^\top \tanh(V_f(t))  + \sum \log \frac{|\cosh(V_f(t+1))|}{|\cosh(V_f(t))|}  - \frac{1}{2} \tanh(V_f(t+1))^\top \Xi \Xi^\top \tanh(V_f(t+1)) + \frac{1}{2} \tanh(V_f(t))^\top \Xi \Xi^\top \tanh(V_f(t)) + \tanh(V_f(t))^\top \Xi \Phi^\top \Xi^\top \tanh(V_f(t))
\end{dmath}
%
\begin{dmath}
	= \frac{1}{2} \left( \tanh(V_f(t))^\top - \tanh(V_f(t+1)^\top) \right) \, \Xi \, \Xi^\top \tanh(V_f(t+1)) + \frac{1}{2} \tanh(V_f(t))^\top \Xi \Xi^\top \left( \tanh(V_f(t)) + \tanh(V_f(t+1)^\top) \right) - V_f(t)^\top \tanh(V_f(t))  + \sum \log \frac{|\cosh(V_f(t+1))|}{|\cosh(V_f(t))|} + \tanh(V_f(t))^\top \Xi \Phi^\top \Xi^\top \tanh(V_f(t))
\end{dmath}

\section{Proof using Concave Convex Procedure}
%
\begin{dmath}
V_f(t+1) = \Xi \, \Xi^\top \, \sigma_f(V_f(t)) + \Xi \, \Phi^\top \, \Xi^\top V_{d}(t)
\end{dmath}
%
\begin{empheq}{equation}
	E = V_f^\top \sigma_f(V_f) - L_f + \frac{1}{2} \sigma_f(V_f)^\top \, \Xi \, \Xi^\top \, \sigma_f(V_f) + \sigma_f(V_f)^\top \, \Xi \Phi^\top \Xi^\top V_d + \frac{1}{2} V_d^\top \Xi \Phi \Phi^\top \, \Xi^\top V_{d}
\end{empheq}
%
Let $x = V_f and y = V_d, \sigma = \sigma_f = \pdv{L}{x}, L = L_f$
%
\begin{dmath}
x(t+1) = \Xi \, \Xi^\top \, \sigma_f(x(t)) + \Xi \, \Phi^\top \, \Xi^\top y(t)
\end{dmath}
%
\begin{empheq}{equation}
	E(x) = x^\top \sigma(x) - L + \frac{1}{2} \sigma(x)^\top \, \Xi \, \Xi^\top \, \sigma(x) + \sigma(x)^\top \, \Xi \, \Phi^\top \Xi^\top y + \frac{1}{2} y^\top \Xi \, \Phi \, \Phi^\top \, \Xi^\top y
\end{empheq}
%
Assume that $E$ is bounded \textcolor{red}{think of proving this}

\subsection*{Function convexity}

\subsubsection*{Energy Term 1, 2}

\begin{dmath}
	\mathcal{J}(x^\top \sigma(x) - L) = \sigma(x)^\top + x^\top . \mathcal{J}(\sigma(x)) - \mathcal{J}(L)
\end{dmath}
%
\begin{dmath}
	\mathcal{J}(x^\top \sigma(x) - L) =  x^\top . \mathcal{J}(\sigma(x))
\end{dmath}
%
\begin{dmath}
	\mathcal{H}(x^\top \sigma(x) - L) =   \mathcal{J}(\mathcal{J}(\sigma(x))^\top . x)
\end{dmath}
%
Let $\mathcal{J}(\sigma(x)) = (J_1 \, J_2 \, J_3 ...)$. note that $x^\top . \mathcal{H}(\sigma)$ is a tensor product
%
\begin{equation}
    \mathcal{H}(x^\top \sigma(x) - L) = \begin{pmatrix}
        J_1^\top + x^\top . \mathcal{H}(\sigma) \\
        J_2^\top + x^\top . \mathcal{H}(\sigma) \\
        \vdots 
    \end{pmatrix} = \mathcal{J}(\sigma(x))^\top + x^\top . \mathcal{H}(\sigma)
\end{equation}
%
\begin{equation}
    \mathcal{H}(x^\top \sigma(x) - L) = \mathcal{H}(L)^\top + \textcolor{red}{x^\top . \mathcal{H}(\sigma)}
\end{equation}
%
\noindent\rule{8cm}{0.4pt} %%%%%%%%%%%%%%%%%%%%%%%%%%%%%%%% \\

\subsubsection*{Energy Term 3}

\begin{dmath}
	\frac{1}{2} \mathcal{J}(\sigma(x)^\top \, \Xi \, \Xi^\top \, \sigma(x)) = \frac{1}{2} \, 2 \, \Xi \, \Xi^\top . \sigma(x) . \mathcal{J}(\sigma) = \sigma(x)^\top \Xi \, \Xi^\top . \mathcal{J}(\sigma)
\end{dmath}
%
\begin{dmath}
	\frac{1}{2} \mathcal{H}(\sigma(x)^\top \, \Xi \, \Xi^\top \, \sigma(x)) = \mathcal{J}((\sigma(x)^\top \Xi \, \Xi^\top . \mathcal{J}(\sigma))^\top)
\end{dmath}
%
\begin{dmath}
	\frac{1}{2} \mathcal{H}(\sigma(x)^\top \, \Xi \, \Xi^\top \, \sigma(x)) = \mathcal{J}(\mathcal{J}(\sigma)^\top \Xi \, \Xi^\top \, \sigma(x))
\end{dmath}
%
\begin{dmath}
	\frac{1}{2} \mathcal{H}(\sigma(x)^\top \, \Xi \, \Xi^\top \, \sigma(x)) = \mathcal{J}(\sigma)^\top \Xi \, \Xi^\top \, \mathcal{J}(\sigma) + \mathcal{J}(\mathcal{J}(\sigma)^\top \Xi \, \Xi^\top) \, \sigma(x)
\end{dmath}
%
\begin{dmath}
	\frac{1}{2} \mathcal{H}(\sigma(x)^\top \, \Xi \, \Xi^\top \, \sigma(x)) = \mathcal{J}(\sigma)^\top \Xi \, \Xi^\top \, \mathcal{J}(\sigma) + \textcolor{red}{\mathcal{H}(\sigma)^\top . \Xi \, \Xi^\top \, \sigma(x)}
\end{dmath}
%
\noindent\rule{8cm}{0.4pt} %%%%%%%%%%%%%%%%%%%%%%%%%%%%%%%% \\

\subsubsection*{Energy Term 4}

\begin{dmath}
    \mathcal{J}(\sigma(x)^\top \, \Xi \, \Phi^\top \Xi^\top y) = y^\top \, \Xi \, \Phi \, \Xi^\top \, \mathcal{H}(\sigma)
\end{dmath}

\end{document}